\documentclass{exam}[letterpaper,10pt]

\usepackage{amsmath}
\usepackage{mathptmx}
\usepackage{helvet}
\usepackage{multicol}
\usepackage{siunitx}
\usepackage[margin=0.5in]{geometry}

\title{Cap\'itulo I: Examen}
\author{...}
\date{Noviembre del 2024}

\begin{document}

{\Large \textsf{\textbf{Cap\'itulo I} Examen}}\\[-0.5\baselineskip]
\rule{1\linewidth}{1pt}
\begin{multicols}{2}
\begin{questions}
    \bf\question
    \begin{parts}
        \part \textnormal{Grafique los intervalos \((-5, 3]\) y \((2, \infty)\) sobre la recta de n\'umeros reales.}
        \part \textnormal{Exprese las desigualdades \(x\leq 3\) y \(-1\leq x < 4\) en notaci\'on de intervalos.}
        \part \textnormal{Encuentre la distancia entre -7 y 9 en la recta de n\'umeros reales.}
    \end{parts}
    \question \textnormal{Eval\'ue cada una de las expresiones siguientes.*}
    \begin{parts}
        \part \((-3)^4\)
        \part \(-3^4\)
        \part \(3^{-4}\)
        \part \(\displaystyle\frac{\displaystyle 5^{23}}{\displaystyle 5^{21}}\)
        \part \(\left(\displaystyle\frac{\displaystyle 2}{\displaystyle 3}\right)^{-2}\)
        \part \(16^{-3/4}\)
    \end{parts}
    \question \textnormal{Escriba cada uno de estos n\'umeros en notaci\'on cient\'ifica.*}
    \begin{parts}
        \part \textnormal{186,000,000,000}
        \part \textnormal{0.0000003965}
    \end{parts}
    \question \textnormal{Simplifique cada expresi\'on. Escriba su respuesta final sin exponentes negativos.*}
    \begin{parts}
        \part \(\sqrt{200}\;-\;\sqrt{32}\)
        \part \((3a^3b^3)(4ab^2)^2\)
        \part \(\displaystyle\left(\frac{\displaystyle 3x^{3/2}y^3}{\displaystyle x^2y^{-1/2}}\right)^{-2}\)
        \part \(\displaystyle\frac{\displaystyle x^2\;+\;3x\;+\;2}{\displaystyle x^2\;-\;x\;-\;2}\)
        \part \(\displaystyle\frac{\displaystyle x^2}{\displaystyle x^2\;-\;4}\,-\,\displaystyle\frac{\displaystyle x\;+\;1}{\displaystyle x\;+\;2}\)
        \part \(\displaystyle\frac{\displaystyle \displaystyle\frac{\displaystyle y}{\displaystyle x}\,-\,\displaystyle\frac{\displaystyle x}{\displaystyle y}}{\displaystyle \displaystyle\frac{\displaystyle 1}{\displaystyle y}\,-\,\displaystyle\frac{\displaystyle 1}{\displaystyle x}}\)
    \end{parts}
    \question \textnormal{Racionalice el denominador y simplifique: \(\displaystyle\frac{\displaystyle\sqrt{10}}{\displaystyle\sqrt{5}\;-\;2}\)}
    \question \textnormal{Realice las operaciones indicadas y simplifique.*}
    \begin{parts}
        \part \(3(x\;+\;6)\;+\;4(2x\;-\;5)\)
        \part \((x\;+\;3)(4x\;-\;5)\)
        \part \((\sqrt{a}\;+\;\sqrt{b})(\sqrt{a}\;-\;\sqrt{b})\)
        \part \((2x\;+\;3)^2\)
        \part \((x\;+\;2)^2\)
    \end{parts}
    \question \textnormal{Factorice por completo cada expresi\'on.*}
    \begin{parts}
        \part \(4x^2\;-\;25\)
        \part \(2x^2\;+\;5x\;-\;12\)
        \part \(x^3\;-\;3x^2\;-\;4x\;+\;12\)
        \part \(x^4\;+\;27x\)
        \part \(3x^{3/2}\;-\;9x^{1/2}\;+\;6x^{-1/2}\)
        \part \(x^3y\;-\;4xy\)
    \end{parts}
    \question \textnormal{Encuentre todas las soluciones reales.*}
    \begin{parts}
        \part \(x\;+\;5\,=\,14\;-\;\frac{1}{2}x\)
        \part \(\displaystyle\frac{\displaystyle 2x}{\displaystyle x\;+\;1}\,=\,\displaystyle\frac{\displaystyle 2x\;-\;1}{\displaystyle x}\)
        \part \(x^2\;-\;x\;-\;12\,=\,0\)
        \part \(2x^2\;+\;4x\;+\;1\,=\,0\)
        \part \(\sqrt{3\;-\;\sqrt{x\;+\;5}}\,=\,2\)
        \part \(x^4\;-\;3x^2\;+\;2\,=\,0\)
        \part \(3\,|\,x\;-\;4\,|\;=\,10\)
    \end{parts}
    \question \textnormal{Mary viaj\'o en auto de Amity a Belleville a una velocidad de 50 mi/h. En el viaje de regreso, manej\'o a 60 mi/h. El total del viaje dur\'o \(4\frac{2}{5}\;\si{h}\) de tiempo de manejo. Encuentre la distancia entre estas dos ciudades.}
    \question \textnormal{Una parcela rectangular de tierras mide 70 pies m\'as larga que su ancho. Cada diagonal entre esquinas opuestas mide 130 pies. ¿Cu\'ales son las dimensiones de la parcela?}
    \question \textnormal{Resuelva estas desigualdades. Escriba la respuesta usando notaci\'on de intervalos y trace la soluci\'on en la recta de n\'umeros reales.*}
    \begin{parts}
        \part \(-4<5\;-\;3x\,\leq\,17\)
        \part \(x(x\;-\;1)(x\;+\;2)\,>\,0\)
        \part \(|\,x\;-\;4\,|\;<\,3\)
        \part \(\displaystyle\frac{\displaystyle 2x\;-\;3}{\displaystyle x\;+\;1}\,\leq\,1\)
    \end{parts}
    \question \textnormal{Se ha de almacenar una botella de medicina a una temperatura entre 5°C y 10°C. ¿A qu\'e intervalo corresponde esto en la escala Fahrenheit? [\emph{Nota:} Las temperaturas Fahrenheit (\emph{F}) y Celsius (\emph{C}) satisfacen la relaci\'on \(C\,=\,\frac{5}{9}(F\;-\;32)\).]}
    \question \textnormal{¿Para qu\'e valores de \(x\) est\'a definida la expresi\'on \(\sqrt{6x\;-\;x^2}\) como un n\'umero real?}
    \question \textnormal{Resuelva gr\'aficamente la ecuaci\'on y la desigualdad.* [\emph{Nota:} Use una calculadora gr\'afica.]}
    \begin{parts}
        \part \(x^3\;-\;9x\;-\;1\,=\,0\)
        \part \(x^2\;-\;1\leq\;|\,x\;+\;1\,|\)
    \end{parts}
    \question
    \begin{parts}
        \part \textnormal{Localice los puntos \(P(0,3)\), \(Q(3,0)\) y \(R(6,3)\) en el plano de coordenadas. ¿D\'onde debe estar ubicado el punto \(S\) para que \(PQRS\) sea un cuadrado?}
        \part \textnormal{Encuentre el \'area de \(PQRS\).}
    \end{parts}
    \question
    \begin{parts}
        \part \textnormal{Trace la gr\'afica de \(y\,=\,x^2\;-\;4\).}
        \part \textnormal{Encuentre los puntos de interesecci\'on del eje \(x\) y \(y\) de la gr\'afica.}
        \part \textnormal{¿La gr\'afica es sim\'etrica alrededor del eje \(x\), del eje \(y\) o del origen?}
    \end{parts}
    \question \textnormal{Sean \(P(-3,1)\) y \(Q(5,6)\) dos puntos en el plano de coordenadas.}
    \begin{parts}
        \part \textnormal{Localice \(P\) y \(Q\) en el plano de coordenadas.}
        \part \textnormal{Encuentre la distancia entre \(P\) y \(Q\).}
        \part \textnormal{Encuentre encuentre el punto medio del segmento \(PQ\).}
        \part \textnormal{Encuentre la pendiente de la recta que contenga a \(P\) y \(Q\).}
        \part \textnormal{Encuentre el bisector perpendicular de la recta que contenga a \(P\) y \(Q\).}
        \part \textnormal{Encuentre la ecuaci\'on para la circunferencia para el que el segmento \(PQ\) es un di\'ametro.}
    \end{parts}
    \question \textnormal{Encuentre el centro y radio de cada circunferencia y trace su gr\'afica.*}
    \begin{parts}
        \part \(x^2\;+\;y^2\,=\,25\)
        \part \((x\;-\;2)^2\;+\;(y\;+\;1)^2\,=\,9\)
        \part \(x^2\;+\;6x\;+\;y^2\;-\;2y\;+\;6\,=\,0\)
    \end{parts}
    \question \textnormal{Escriba una ecuaci\'on lineal \(2x\;-\;3y\,=\,15\) en forma de pendiente e intersecci\'on, y trace su gr\'afica. ¿Cu\'ales son la pendiente y el punto de intersecci\'on \(y\)?}
    \question \textnormal{Encuentre una ecuaci\'on para la recta con la propiedad dada.}
    \begin{parts}
        \part \textnormal{Pasa por el punto \((3,-6)\) y es paralela a la recta \(3x\;+\;y\;-\;10\,=\,0\).}
        \part \textnormal{Tiene punto de intersecci\'on \(x\) en 6 y punto de intersecci\'on \(y\) en 4.}
    \end{parts}
    \question \textnormal{Un ge\'ologo usa una sonda para medir la temperatura \emph{T} (en °C) del suelo, a varias profundidades debajo de la superficie, y encuentra que a una profundidad de \(x\) cent\'imetros la temperatura est\'a dada por la ecuaci\'on lineal \(T\,=\,0.08x\;-\;4\).}
    \begin{parts}
        \part \textnormal{¿Cu\'al es la temperatura a una profundidad de 1 metro (100 cm)?}
        \part \textnormal{Trace una gr\'afica de la ecuaci\'on lineal.}
        \part \textnormal{¿Qu\'e representan la pendiente, la intersecci\'on en \(x\) y la intersecci\'on en \emph{T} de la gr\'afica de esta ecuaci\'on?}
    \end{parts}
    \question \textnormal{El peso m\'aximo \emph{M} que puede ser soportado por una viga es conjuntamente proporcional a su ancho \emph{w} y el cuadrado de su altura \emph{h}, e inversamente proporcional a su longitud \emph{L}.}
    \begin{parts}
        \part \textnormal{Escriba una ecuaci\'on que exprese esta proporcionalidad.}
        \part \textnormal{Determine la constante de proporcionalidad si una viga de 4 pulg. de ancho, 6 pulg. de alto y 12 pies de largo puede soportar un peso de 4800 libras.}
        \part \textnormal{Si una viga de 10 pies hecha del mismo material mide 3 pulg. de ancho y 10 pulg. de alto, ¿cu\'al es el peso m\'aximo que puede soportar?}
    \end{parts}
\end{questions}
\end{multicols}
\end{document}
